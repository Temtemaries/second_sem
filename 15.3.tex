\documentclass{article}
\usepackage[utf8]{inputenc}

\usepackage[russian]{babel}
\usepackage{amsmath}
\usepackage{hyperref} % Включить ссылки в PDF
\numberwithin{equation}{section} % Изменить нумерацию формул (1) -> (1.1)
\begin{document}
    \tableofcontents % Вставить содержание
    \newpage
    \section{Скорость примитивных БЧХ-кодов}
      Теорема $14.20$ гарантирует существование примитивного БЧХ-кода
      длин

     $n=(2^m)-1$ с конструктивным расстоянием $2t + 1$ и скоростьюома

    \begin{equation}
    R=\frac{n-t\log_2 (n+1)}{n}=1-\frac{t\log_2 (n+1)}{n}
  \label{eq.simple}
    \end{equation}
   Сравним величину $(1.1)$ с границами $(13.13)$ скорости максимального линейного кода. Если $\log_2 t = o(\log_2 n)$, то
    \begin{equation}
    H\left (\frac{t}{n}\right)=-\frac{t}{n}\log_2 \frac{t}{n}-\left(1-\frac{t}{n}\right)\log_2\left(1-\frac{t}{n}\right)=\frac{t}{n}\log_2 n\left(1-\mathfrak{O}\left(\frac{\log_2 t}{\log_2 n}\right)\right) \sim \frac{t}{n}\log_2 n

    \label{eq.second}
    \end{equation}
Поэтому при $n\rightarrow\infty$ и $\log_2 t = o(\log_2 n)$ для скорости примитивных БЧХ-кодов длины $n$, исправляющих $t$ ошибок, справедливо равенство
    \begin{equation}
    R=1-H\left(\frac{t}{n}\right)(1+o(1)),
        \label{eq.3d}
    \end{equation}
    правая часть которого с точностью до слагаемых вида $o(H(t/n))$ равна
верхней оценке скорости максимального линейного кода из $(13.8)$. Таким
образом для малых расстояний БЧХ-коды являются асимптотически мак-
симальными среди всех линейных кодов.
К сожалению, с ростом $t$ ситуация меняется, и скорость БЧХ-кодов ста-
новится меньше не только верхней оценки в $(13.13)$, но и нижней. Рассмот-
рим подробно случай больших относительных расстояний. Из теорем $15.2$ и
$15.3$ следует, что для скорости $R$ примитивного БЧХ-кода длины $n$ с большим минимальным расстоянием $d = 2^p-1$ справедливо асимптотическое равенство
\begin{equation}
    R\sim \left(1-\frac{d_M\varphi(n,d_K)}{4n}\right)^\log_2 n

            \label{equation}
   \end{equation}
   Также нетрудно видеть, что если конструктивное расстояние $d$ примитивного БЧХ-кода удовлетворяет неравенствам $2^{p-1}-1<d\le2^P-1$, то минимальное расстояние этого кода не превосходит $2^p-1$. Следовательно, минимальное расстояние $d_M$ любого примитивного БЧХ-кода не более чем в два раза больше его конструктивного расстояния $d_K$. Поэтому отсюда и из $(1.4)$ следует, что если минимальное расстояние $d_M$ по порядку величины растет быстрее чем $n/\log_2 n$, т. е. $d_M=\psi(n)n/ \log_2 n$, где $\psi(n)\rightarrow\infty$ при
$n\rightarrow\infty$, то
\begin{equation}
    {R\lesssim \left(1-\frac{d_M\varphi(n,d_K)}{4n}\right)^\log_2 n}
    \le {\left(1-\frac{\psi(n)}{4\log_2 n}\right)^\log_2 n}=
    \left(\left(1-\frac{\psi(n)}{4\log_2 n}\right)^{4\log_2 n/ \varphi(n)\right)^{\psi(n)/4}\le{2^{-\psi(n)/4}}.
\end{equation}
Следовательно, если минимальное расстояние примитивного БЧХ-кода дли-
ны $n$ растет быстрее чем $n/\log_2 n$, то его скорость стремится к нулю с ростом $n$. Поэтому при использовании БЧХ-кодов в двоичном симметрич-
ном канале при возрастании длины кода и стремлении к нулю вероятности
неправильного декодирования также к нулю будет стремиться и скорость
передачи информации.
\end{document}
